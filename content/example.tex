% заголовок 1
\section{Заголовок 1 уровня}

% заголовок 2
\subsection{Заголовок 2 уровня}

% заголовок 3
\subsubsection{Заголовок 3 уровня}

% заголовок 
\section{Ещё один заголовок 1 уровня}

% основной текст
Текст текст текст текст текст текст текст текст текст текст текст текст текст текст текст текст текст текст текст текст текст текст текст текст текст текст текст текст текст текст текст текст текст текст текст текст текст текст текст текст текст. Ссылка на литературу \cite{GN-lse}.

Список:

\begin{enumerate}
    \item Предмет 1
    \item Предмет 2
    \item Предмет 3
    \item Предмет 4   
\end{enumerate}

Формула:

\begin{equation} \label{friis-formula}
    \frac{P_r}{P_t} = D_t D_r \left(\frac{\lambda}{4 \pi d}\right) ^ 2,
\end{equation}

Ссылка на формулу \ref{friis-formula}.

Таблица:

\begin{center}
    \vspace{-0.8em}
    \begin{longtable}{|M{8cm}|M{4cm}|}
        \caption{Экспонента затухания для различных сред} \label{PLE-table} \\
        \hline
        Среда                             & Экспонента затухания $n$ \\ \hline
        Свободное пространство            & $2$                      \\ \hline
        Городская местность               & $[2,7; 3,5]$             \\ \hline
        Городская местность с затенением  & $[3; 5]$                 \\ \hline
        Внутри здания (прямая видимость) & $[1,6; 1,8]$             \\ \hline
        Внутри здания с помехами          & $[4; 6]$                 \\ \hline
        Фабрика                           & $[2; 3]$                 \\ \hline
    \end{longtable}
    \vspace{-1.5em}
\end{center}

А вот тут у нас картинка \ref{graph-fig}!!!

\image{content/images/doggy.jpg}{graph-fig}{Рисунок собаки}{0.25}

Ссылаемся на график ~\ref{graph-fig}.

Ещё график \ref{dog-fig}
\image{content/images/doggy.jpg}{dog-fig}{Монти Пайтон, но поменьше}{0.1}


\begin{listing}[ht]
    \inputminted{python}{content/source/fib.py}
    \caption{Fibonacci series}
    \label{lst1}
    \end{listing}


Ссылка на листинг \ref{lst1}.
